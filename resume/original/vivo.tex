%# -*- coding:utf-8 -*-
\documentclass[11pt,a4paper]{moderncv}
\usepackage{fontspec,xunicode}
\setmainfont{Tahoma}
\usepackage[slantfont,boldfont]{xeCJK}

\usepackage{xcolor}       
\setmainfont{Times New Roman}%缺省英文字体.serif是有衬线字体sans serif无衬线字体
\setCJKmainfont[ItalicFont={SimSun}, BoldFont={Microsoft YaHei}]{STSong}%衬线字体 缺省中文字体为
\setCJKsansfont{STSong}
\setCJKmonofont{STFangsong}%中文等宽字体
%-----------------------xeCJK下设置中文字体------------------------------%
\setCJKfamilyfont{song}{SimSun}                             %宋体 song
\newcommand{\song}{\CJKfamily{song}}
\setCJKfamilyfont{fs}{FangSong_GB2312}                      %仿宋2312 fs
\newcommand{\fs}{\CJKfamily{fs}}
\setCJKfamilyfont{yh}{Microsoft YaHei}                    %微软雅黑 yh
\newcommand{\yh}{\CJKfamily{yh}}
\setCJKfamilyfont{hei}{SimHei}                              %黑体  hei
\newcommand{\hei}{\CJKfamily{hei}}
\setCJKfamilyfont{hwxh}{STXihei}                                %华文细黑  hwxh
\newcommand{\hwxh}{\CJKfamily{hwxh}}
\setCJKfamilyfont{asong}{Adobe Song Std}                        %Adobe 宋体  asong
\newcommand{\asong}{\CJKfamily{asong}}
\setCJKfamilyfont{ahei}{Adobe Heiti Std}                            %Adobe 黑体  ahei
\newcommand{\ahei}{\CJKfamily{ahei}}
\setCJKfamilyfont{akai}{Adobe Kaiti Std}                            %Adobe 楷体  akai
\newcommand{\akai}{\CJKfamily{akai}}


%------------------------------设置字体大小------------------------%
\newcommand{\chuhao}{\fontsize{42pt}{\baselineskip}\selectfont}     %初号
\newcommand{\xiaochuhao}{\fontsize{36pt}{\baselineskip}\selectfont} %小初号
\newcommand{\yihao}{\fontsize{28pt}{\baselineskip}\selectfont}      %一号
\newcommand{\erhao}{\fontsize{21pt}{\baselineskip}\selectfont}      %二号
\newcommand{\xiaoerhao}{\fontsize{18pt}{\baselineskip}\selectfont}  %小二号
\newcommand{\sanhao}{\fontsize{15.75pt}{\baselineskip}\selectfont}  %三号
\newcommand{\sihao}{\fontsize{14pt}{\baselineskip}\selectfont}         %四号
\newcommand{\xiaosihao}{\fontsize{12pt}{\baselineskip}\selectfont}  %小四号
\newcommand{\wuhao}{\fontsize{10.5pt}{\baselineskip}\selectfont}    %五号
\newcommand{\subwuhao}{\fontsize{10pt}{\baselineskip}\selectfont}    %次五号
\newcommand{\xiaowuhao}{\fontsize{9pt}{\baselineskip}\selectfont}   %小五号
\newcommand{\liuhao}{\fontsize{7.875pt}{\baselineskip}\selectfont}  %六号
\newcommand{\qihao}{\fontsize{5.25pt}{\baselineskip}\selectfont}    %七号


%\usepackage{fontawesome}
% \setCJKmainfont[BoldFont={WenQuanYi Micro Hei/Bold}]{WenQuanYi Micro Hei}
%\defaultfontfeatures{Mapping=tex-text}
%\XeTeXlinebreaklocale "zh"
%\XeTeXlinebreakskip = 0pt plus 1pt minus 0.1pt
% moderncv themes
\moderncvtheme[blue]{classic}                 % optional argument are 'blue' (default), 'orange', 'red', 'green', 'grey' and 'roman' (for roman fonts, instead of sans serif fonts)
%\moderncvtheme[green]{classic}                % idem
%\moderncvtheme[blue,roman]{hht}
% character encoding

% adjust the page margins
\usepackage[scale=0.9]{geometry}
%\setlength{\hintscolumnwidth}{3cm}						% if you want to change the width of the column with the dates
%\AtBeginDocument{\setlength{\maketitlenamewidth}{6cm}}  % only for the classic theme, if you want to change the width of your name placeholder (to leave more space for your address details
\AtBeginDocument{\recomputelengths}                     % required when changes are made to page layout lengths

% personal data
\firstname{赵尹源}
\familyname{}
\title{Yinyuan Zhao}   
%\address{1990/11/11}{}    % optional, remove the line if not wanted
\mobile{18636994018}                    % optional, remove the line if not wanted
\email{yyzhao@aliyun.com}                     % optional, remove the line if not wanted
%\homepage{Blog: http://geekplux.com} % optional, remove the line if not wanted
%\social[github]{GitHub: https://github.com/geekplux}
\extrainfo{%
%  LinkedIn: https://cn.linkedin.com/in/xxx \\
  WeChat: zyy6993388 \\
  QQ: 527864230}
\photo[64pt]{avatar.jpg}                       % '64pt' is the height the picture must be resized to and 'picture' is the name of the picture file; optional, remove the line if not wanted

\begin{document}
\maketitle
\vspace*{-15mm}

\section{教育经历}
\cventry{12.09-16.07}{本科}{广州大学}{信息安全}{}{}         % arguments 3 to 6 are optional
\cventry{17.09- 至今}{硕士}{西安电子科技大学}{网络空间安全}{云数据安全--数据定位}{}    
\section{项目经历}
\cvline{Geolocation}{云环境下的数据安全定位/验证,解决用户数据所有权和管理权分离问题}
\cvline{ }{算法设计与实现,国家自然科学基金(论文+专利)}
\cvline{ }{密文数据;隐私保护;防止位置伪造;减少带宽开销;分布式}
\cvline{ }{使用Java 实现B/S 架构,服务器实时广播证明,分布式节点合作检测延迟}
\cvline{Kubernetes}{为实验室计算机集群部署容器化应用kubernetes及维护}
\cvline{ }{apiserver,docker,kubelet,dashboard......}
\cvline{Blockchain}{区块链安全性研究,梳理调研区块链各层结构漏洞和威胁}
\cvline{}{密码学与隐私保护、P2P与广播、51\%攻击、合约虚拟机、挖矿木马......}
\cvline{}{智能合约应用于完整性审计+去重,实现惩罚机制}
\section{证书获奖}
\cvline{13.09-17.09}{ 学校奖学金,普通话二级甲等,CET6,IELTS5.5,计算机一级...}
\cvline{17.09-18.09}{ 专业课成绩前10\%、一等奖学金、优秀研究生称号 }    
\cvline{18.04-18.06}{ 第四届“互联网+创新创业大赛”金奖:基于区块链支持模糊去重的完整性审计系统}
\cvline{}{数据加密+重复数据删除+数据完整性验证+区块链智能合约}      
\cvline{18.10-19.01}{ 研究生学术年会论文二等奖 }
\section{工作技能}
\cvline{基础}{无障碍阅读英文;密码学基础;云存储安全;区块链基础;实现论文方案的能力}
\cvline{}{熟练使用排版工具Office、Markdown、LaTex}
\cvline{}{网络原理和安全,熟悉HTTP(S)、TCP/IP}
\cvline{前端}{了解HTML、CSS}
\cvline{后端}{熟悉常用的命令行工具,熟练掌握Shell、Python、C语言编程}
\section{科研成果}
\cvline{论文,已发表}{\textbf{Zhao, Y.,} Yuan, H., Jiang, T., Chen, X., \href{https://doi.org/10.1016/j.jisa.2019.04.003} {Secure Distributed Data Geolocation Scheme Against Location Forgery Attack}, \emph{Journal of Information Security and Applications}, (CCF C类), (SCI)}
\cvline{论文,已录用}{Hua, M., \textbf{Zhao, Y.,} Jiang, T., Secure Data Deletion in Cloud Storage: A Survey,  \emph{International Journal of Embedded Systems}}
\cvline{专利,实审中}{姜涛,\textbf{赵尹源},袁浩然,王一凡,基于完整性审计与通信时延的云数据安全定位方法}
\section{自我评价}
\cvline{技术}{熟练使用Python语言进行开发,应用所学知识,有良好的设计思路}
\cvline{性格}{学习能力和对环境的适应能力强,分析能力强,有责任心,善于沟通交流}
\cvline{兴趣}{球类运动;贴吧吧主管理}
\end{document}