\documentclass[12pt]{test} 
\usepackage{ctex}
\usepackage{fontawesome}
\setmainfont{SimSun}
\newfontfamily{\FA}{FontAwesome.otf}
\begin{document}
\begin{minipage}[t]{0.45\textwidth} % 45% of the page width for name
	\vspace{-\baselineskip} % Required for vertically aligning minipages
	{{\HUGE{\textbf{\MakeUppercase{赵尹源}}}}}
	\vspace{3pt}
	
	{申请职位:运维工程师} % Career or current job title
\end{minipage}
\begin{minipage}[t]{0.26\textwidth} % 27.5% of the page width for the first row of icons
	\vspace{-\baselineskip} % Required for vertically aligning minipages
	\faMobilePhone   {  18636994018}\\
	\faWeixin   {  zyy6993388}\\
	
\end{minipage}
\begin{minipage}[t]{0.275\textwidth} % 27.5% of the page width for the second row of icons
	\vspace{-\baselineskip} % Required for vertically aligning minipages
	
	% The first parameter is the FontAwesome icon name, the second is the box size and the third is the text
	% Other icons can be found by referring to fontawesome.pdf (supplied with the template) and using the word after \fa in the command for the icon you want
	\faMailForward   {  \href{mailto:yyzhao@aliyun.com}{yyzhao@aliyun.com}}\\	
	\faQq   {  527864230}\\
\end{minipage}

\vspace{0.5cm}

\cvsect{工作技能}
\begin{minipage}[t]{0.5\textwidth} % 40% of the page width for the introduction text
% 	\vspace{-\baselineskip} % Required for vertically aligning minipages
		\bubbles{6/LaTex,4/Office,3/HTML\quad,5/Markdown}	
\end{minipage}
\hfill % Whitespace between
\begin{minipage}[t]{0.6\textwidth} % 50% of the page for the skills bar chart
\begin{barchart}{6.5}
		\baritem{Python}{85}
		\baritem{SQL}{65}
		\baritem{Git}{80}
		\baritem{Shell}{90}
		\baritem{Java}{45}
	\end{barchart}
\end{minipage}
\vspace{0.5cm}

\cvsect{项目经历}

\begin{entrylist}
		\vspace{6pt}
	\entry	{Geolocation}{云环境下的数据安全定位/验证}	{国家自然科学基金\\}{用Java 实现B/S 架构的云存储系统,服务器实时广播证明,分布式节点合作检测延迟\\		\texttt{密文数据} \slashsep\texttt{隐私保护} \slashsep\texttt{防止位置伪造} \slashsep\texttt{减少带宽开销} \slashsep\texttt{分布式}}
		\vspace{6pt}
	\entry		{Kubernetes}	{实验室计算机集群部署容器化应用kubernetes及维护}		{实验室内部平台\\}	{\texttt{apiserver} \slashsep\texttt{docker} \slashsep\texttt{kubelet} \slashsep\texttt{dashboard}}
	\entry	{Blockchain}	{区块链安全性研究,各层结构漏洞和威胁}	{丛书出版中\\}		{\texttt{密码学与隐私保护} \slashsep\texttt{P2P与广播} \slashsep\texttt{51\%攻击} \slashsep\texttt{合约虚拟机} \slashsep\texttt{挖矿木马}}
			\vspace{1pt}
\end{entrylist}
\cvsect{教育经历}
\begin{entrylist}
\entry{12.09-16.07}	{广州大学:数学与信息科学学院:信息安全}	{本科}{}
\entry{17.09- 至今}{西安电子科技大学:网络空间安全:云数据安全--数据定位}{硕士}{}
			\vspace{1pt}
\end{entrylist}

\cvsect{成果获奖}
\begin{entrylist}
\entry{13.09-17.09}{学校奖学金,普通话二级甲等,CET6,IELTS5.5...}{}{}
\entry{17.09-18.09}{专业课成绩前10\%、一等奖学金、优秀研究生称号 }	{}{}
\entry{18.04-18.06}{第四届“互联网+创新创业大赛”金奖}{}{}
\entry{18.10-19.01}{ 研究生学术年会论文二等奖 }{}{}
\end{entrylist}
			\vspace{6pt}
\begin{entrylist}
\entry{论文,已发表}{\textbf{Zhao, Y.,} Yuan, H., Jiang, T., Chen, X., \href{https://doi.org/10.1016/j.jisa.2019.04.003} {Secure Distributed Data Geolocation Scheme Against Location Forgery Attack}, \emph{Journal of Information Security and Applications}, (CCF C类), (SCI)}{}{}
\entry{论文,已录用}{Hua, M., \textbf{Zhao, Y.,} Jiang, T., Secure Data Deletion in Cloud Storage: A Survey,  \emph{International Journal of Embedded Systems}}{}{}
\entry{专利,实审中}{姜涛,\textbf{赵尹源},袁浩然,王一凡,基于完整性审计与通信时延的云数据安全定位方法}{}{}
\end{entrylist}
\end{document}